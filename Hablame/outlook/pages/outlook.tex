\section{Ausblick und Anregungen}
	Hier sind Anregungen für die Weiterentwicklung und mögliche Zukunftszenarien zu finden:\\	
	\textbf{Backend}
	\begin{itemize}\itemsep0pt
		\item{TLS Zertifikat für den Webserver und den Tomcatserver}
		\item{Mehr Extensions nutzen um externes Wissen einzubinden (wie bei der Wetterabfrage schon geschehen)}
		\item{Mehrbenutzerbetrieb (gehaltenes Wissen des Bots pro User)}
	\end{itemize}
	\textbf{Service-API}
	\begin{itemize}\itemsep0pt
		\item{Die Conversation-Klasse kann genutzt werden um eine Gesprächshistorie einzufügen}
	\end{itemize}	
	\textbf{Android Application}
	\begin{itemize}\itemsep0pt
		\item{Einstellungsmenü erstellen}
		\item{Funktion \textit{OnListenForName} im Einstellungsmenü ein/ausschaltbar}
		\item{Sprachwahl im Einstellungsmenü}
		\item{Sprachausgabe-Stimme im Einstellungsmenü wählbar (weiblich/männlich)}
		\item{Mehrere Designs; im Einstellungsmenü nach Belieben wählbar}
	\end{itemize}	
	\textbf{Allgemeines}
	\begin{itemize}\itemsep0pt
		\item{Pädagogenschnittstelle: Eigenständige Software, die die AIML-Datein parst und Einträge bearbeiten kann}
	\end{itemize}