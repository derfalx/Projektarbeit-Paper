\section{Überarbeitung der Server-Architektur}
	Im Rahmen der Projektarbeit soll die serverseitige Architektur überarbeitet
	und mit der Nutzung von modernen Tools und Frameworks ausgestattet werden.
	Diese Optimierungen werden in den folgenden Unterkapiteln beschrieben.
	
	\subsection{Datenbank}
		Eine Anforderung an das Projekt war es, dass die bisherige Nutzung einer
		Oracle Datenbank abgelöst wird durch eine frei verfügbare Datenbank. Das Team
		entschied sich für das freie relationale Datenbankmanagementsystem MySQL in
		der aktuellsten produktiven Version 5.6.26 zum Zeitpunkt dieser Arbeit.
		Dies bringt den großen Vorteil mit sich, eine weitverbreitete quelloffene, statt
		einer proprietären Datenbank Software zu nutzen.
		
	\subsection{Das Spring Framework}
		Eine weitere Anforderung stellt die Nutzung moderner Frameworks und
		Technologien dar. Um einerseits einen guten Programmierstil sicher stellen zu
		können und andererseits die Kollaboration zu verbessern, sowie zukünftigen
		Projektteams den Einstieg maßgeblich zu erleichtern.
		Spring ermöglicht es durch so genannte Annotationen die Persistierung von
		Objekten unter Zuhilfenahme von Hibernate mit Object-Relational-Mapping zu
		vereinfachen. Zusätzlich wird das Testen mit Unittests durch JUnit
		simplifiziert und die Instanziierung von Objekten zur Laufzeit im
		Singleton-Pattern durchgeführt. Das Speichern und Lesen von Objekten kann
		unter Spring beispielsweise durch Interfaces mit der @Respository Annotation
		übernommen werden. So ist es möglich ohne eine SQL Anweisung selbst zu
		schreiben, Datenbankzugriffe zu realisieren und mit Objekten zu arbeiten.
		Dies erleichtert das Erstellen der Daten-Zugriffsschicht enorm. Diese
		Fähigkeit wird durch das Beerben des Interfaces CrudRepository, oder eines
		davon erbenden Interfaces erworben. Das Beerben eines dieser Interfaces
		verpflichtet zu den grundlegenden Datenbankoperationen Create, Read, Update
		und Delete eines spezifischen Objekttyps. Somit können POJOs in der
		Datenbank direkt abgespeichert werden. Diese werden in Form von separaten
		Model-Klassen angelegt, mit der @Entity Annotation versehen und sind somit
		als Datenbank-Entität gekennzeichnet. Weiterhin kann der Tabellenname, sowie
		einzelne Spaltennamen angegeben werden und Kardinalitäten zu anderen POJOs
		festgelegt werden, z.B. hat eine Kategorie mehrere Themengebiete.\\
		Die Spring Hierarchie ist ein wesentlicher Bestandteil der neu designten
		Server-Architektur und soll hier deshalb zusätzlich Einzug erhalten. Wie
		bereits erwähnt können einzelne Interfaces mit der @Repository Annotation für
		ORM genutzt werden. Diese Datenbankschicht der Hierarchie ist die unterste
		der drei und wird von der darüber befindlichen Service-Ebene genutzt. Hier
		werden Klassen mit der @Service Annotation zum Service ermächtigt, was dem
		Spring Kontext mitteilt, dass diese Services zur Laufzeit als Singleton
		Instanziiert werden. Services nutzen also die darunter liegenden Repositories
		für den Datenbankzugriff und verarbeiten die Daten weiter, bzw. führen andere
		gewünschte Operationen aus. Services stellen also Logikschicht dar. Diese
		Logikschicht wird von der API, den sogenannten Controllern genutzt, um
		Anfragen, sogenannte Web-Requests in Form eines GET oder POST, weiter
		bearbeiten zu können. Ein Controller hält die @Autowired Annotierten Services
		und dieser wiederum die Repositories und diese Objekte werden zur Laufzeit
		Instanziiert.