%Papierformat, mit Koma-Script Dokumentenklasse (Europäisches Design)
\documentclass[12pt,a4paper]{scrartcl}
%Deutsche Silbentrennung
\usepackage[ngerman]{babel}
%Listen einrücken
\usepackage{enumitem}
%Deutsche Umlaute
\usepackage[utf8]{inputenc}
%Trennung von deutschen Umlauten
\usepackage[T1]{fontenc}
%Bibtex für Zitate,
% see: https://en.wikibooks.org/wiki/LaTeX/Bibliography_Management#Customization
\usepackage[]{natbib}
%Grafikpaket laden
\usepackage{graphicx}
%Grafik Floating einschränken
\usepackage{placeins}
%Referenzierung mit Name
\usepackage{titleref}
%Unterschritszeilen
\usepackage{tabularx}
%Verlinkung
% see:https://en.wikibooks.org/wiki/LaTeX/Hyperlinks
\usepackage[hidelinks]{hyperref}
%Urls sollen erkannt werden
\usepackage{url}
%Ermöglicht es von Schrift umflossene Bilder und Tabellen einzufügen
\usepackage{wrapfig}
%Sourcecode anzeigen lassen
\usepackage{listings}
%Farbenpaket laden
\usepackage{xcolor}
%Einrichten des Linking
%\hypersetup{
%    colorlinks,
%    linkcolor={red!50!black},
%    citecolor={blue!50!black},
%    urlcolor={blue!80!black}
%}
%Umbennen der Bibliography Angabe in Literatur
\renewcommand{\bibname}{Literatur}
%Festlegung des Zitierstiles
\bibliographystyle{plain}
%Inhaltsverzeichnis Tiefe erweitern
\setcounter{tocdepth}{5}
%Nummerierung Tiefe erweitern
\setcounter{secnumdepth}{5}

%\makeatletter
%\newcommand\footnoteref[1]{\protected@xdef\@thefnmark{\ref{#1}}\@footnotemark}
%\makeatother​

%Dokumentbeginn
\begin{document}
	
	\begin{titlepage}
	%Eine mbox wird verwendet um Text zusammenzuhalten
	%vspace erzeugte die in Klammern angegebenen Zeilenabstände
	%baselineskip setzt zeilenabstand
   	\mbox{}\vspace{5\baselineskip}\\
   	%Schriftart und Größe als Attribut
   	\rmfamily\huge
   	%Mittige Textausrichtung (\centerline für eine Zeile)
   	\centering
   	%Das Argument erscheint in Kapitaelchen (small capitals).
	\textsc{Weiterentwicklung von Hablame}
	%Umbruch bezogen auf die Hoehe des Kleinbuchstaben x in diesem Element * Faktor
	\\[3ex]
   	Projektarbeit
   	\rmfamily\Large
   	\vspace{1\baselineskip}\\
   	%Externes einbinden einer Textdatei
   	%% versionsnummer entfernt
   	%\input{version.txt}\mbox{}
	\vspace{3\baselineskip}
	Hochschule für angewandte Wissenschaften Würzburg-Schweinfurt
   	\vspace{5\baselineskip}\\
   	\rmfamily\Large
   	David Artmann\\
   	\rmfamily\Large
   	Dominik Hirsch\\
   	\rmfamily\Large
   	Kristoffer Schneider
   	\vspace{1\baselineskip}\\
   	%Heutiges Datum
   	\today
\end{titlepage}

	%Inhaltsverzeichnis
	\tableofcontents
	\newpage
	
	%Abbildungsverzeichnis
	\listoffigures
	

	%\newpage
	%\bibliography{general/bibtex/bibtex.bib}
	%Bibliography im Inhaltsverzeichnis anzeigen(muss unterhalb von bibliography
	% sein)
	\addcontentsline{toc}{section}{Literatur}
\end{document}
