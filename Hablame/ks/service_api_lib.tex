\section{Die Service-API-Library}
Um die Nutzung des Backends zu vereinfachen wurde ein weiteres Teilprojekt erstellt: die Service-API-Library. Dieses Projekt bietet einen einfach zu handhabenden Client zur Nutzung des Bot-Services. Dafür bietet dieser Methoden zur jeder Schnittstelle des Backends.\\
Durch dieses Vorgehen ergibt sich der Vorteil, dass somit eine einheitliche, leicht zu nutzende Schnittstelle für Nutzerprogramme geboten wird, welche auch in weiteren Anwendungen integriert werden kann (z.B. in eine Desktopanwendung).\\
Als Basis für die Rest-Kommunikation wird die Bibliothek UniREST von Mashape verwendet.\footnote{Unirest: http://unirest.io/}